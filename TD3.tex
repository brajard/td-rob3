\documentclass[french,12pt,a4paper,twoside,openright,titlepage]{report}
\usepackage{algorithm}
\usepackage{algorithmic}
\usepackage{ucs}
\usepackage[utf8x]{inputenc}
\usepackage{fancyhdr}
\usepackage{amsmath}
\usepackage{amssymb}
\usepackage{babel}
\usepackage{amsfonts}
\usepackage{graphicx}
%\usepackage{here}
\usepackage{ifthen}
\usepackage{xcomment}
\usepackage{array}
\usepackage{rotating}

\textwidth17cm
\textheight25cm
%\oddsidemargin-1cm
%\parskip0.2pt
\voffset-2cm
\hoffset-1cm
\headsep0.5cm
\topmargin0.5cm
\oddsidemargin0.5cm
\evensidemargin0.5cm

\makeatletter
\@addtoreset{section}{part}
\makeatother
\newlength{\mylength}
\newcounter{exo}\stepcounter{exo}
\newcommand\exercice[1]{
\subsubsection*{Exercice \theexo\stepcounter{exo} {#1}}
%{\bf Exercice \theexo\stepcounter{exo}\\}
}


\RequirePackage{layout,color}

\definecolor{gray50}{gray}{.5}
\definecolor{gray40}{gray}{.6}
\definecolor{gray30}{gray}{.7}
\definecolor{gray20}{gray}{.8}
\definecolor{gray10}{gray}{.9}
\definecolor{gray05}{gray}{.95}

\newsavebox{\laboiboite}
\newlength{\longdelaboiboite}
\newlength{\deptdelaboiboite}
\newlength{\hautdelaboiboite}
\newcolumntype{R}{>{\begin{lrbox}{\laboiboite} 
$\displaystyle }r<{$\end{lrbox}%
    \settowidth{\longdelaboiboite}{\usebox{\laboiboite}}%
    \settoheight{\hautdelaboiboite}{\usebox{\laboiboite}}%
    \settodepth{\deptdelaboiboite}{\usebox{\laboiboite}}%
    \addtolength{\hautdelaboiboite}{\deptdelaboiboite}%
    \addtolength{\hautdelaboiboite}{2ex}%
    \parbox[c][\hautdelaboiboite][c]{0cm}{} \makebox[\width]
{\usebox{\laboiboite}}}}
\newcolumntype{C}{>{\begin{lrbox}{\laboiboite} 
$\displaystyle }c<{$\end{lrbox}%
    \settowidth{\longdelaboiboite}{\usebox{\laboiboite}}%
    \settoheight{\hautdelaboiboite}{\usebox{\laboiboite}}%
    \settodepth{\deptdelaboiboite}{\usebox{\laboiboite}}%
    \addtolength{\hautdelaboiboite}{\deptdelaboiboite}%
    \addtolength{\hautdelaboiboite}{2ex}%
    \parbox[c][\hautdelaboiboite][c]{0cm}{} \makebox[\width]
{\usebox{\laboiboite}}}}
\newcolumntype{L}{>{\begin{lrbox}{\laboiboite} 
$\displaystyle }l<{$\end{lrbox}%
    \settowidth{\longdelaboiboite}{\usebox{\laboiboite}}%
    \settoheight{\hautdelaboiboite}{\usebox{\laboiboite}}%
    \settodepth{\deptdelaboiboite}{\usebox{\laboiboite}}%
    \addtolength{\hautdelaboiboite}{\deptdelaboiboite}%
    \addtolength{\hautdelaboiboite}{2ex}%
    \parbox[c][\hautdelaboiboite][c]{0cm}{} \makebox[\width]
{\usebox{\laboiboite}}}}

\newcolumntype{S}{>{\begin{lrbox}{\laboiboite}}r<{\end{lrbox}%
    \mbox{\usebox{\laboiboite}}}}
\newcolumntype{D}{>{\begin{lrbox}{\laboiboite}}c<{\end{lrbox}%
    \mbox{\usebox{\laboiboite}}}}
\newcolumntype{M}{>{\begin{lrbox}{\laboiboite}}l<{\end{lrbox}%
    \mbox{\usebox{\laboiboite}}}}
\newlength\Linewidth
\def\findlength{\setlength\Linewidth\linewidth
\addtolength\Linewidth{-4\fboxrule}
\addtolength\Linewidth{-3\fboxsep}
}
\newenvironment{examplebox}{\par\begingroup%
   \setlength{\fboxsep}{5pt}\findlength%
   \setbox0=\vbox\bgroup\noindent%
   \hsize=\Linewidth%
   \begin{minipage}{\Linewidth}\small}%
    {\end{minipage}\egroup%
    \vspace{6pt}%
    \noindent\textcolor{gray20}{\fboxsep2.5pt\fbox%
     {\fboxsep5pt\colorbox{gray05}{\normalcolor\box0}}}%
    \endgroup\par\addvspace{6pt minus 3pt}\noindent%
    \normalcolor\ignorespacesafterend}
\let\Examplebox\examplebox
\let\endExamplebox\endexamplebox


%%%%%%%%%%%%%%%%%%%%%%%%%%%%%%%%%%%%%%%%%%%%%%%%%%%%%%%%%%%%%%
%%%%%%%%%%%%%%%%%%%%%%%%%%%%%%%%%%%%%%%%%%%%%%%%%%%%%%%%%%%%%%
%%%%%%%%%%%%%%%%%%%%%%%%%%%%%%%%%%%%%%%%%%%%%%%%%%%%%%%%%%%%%%
\newboolean{sol}
%\setboolean{sol}{true}
\setboolean{sol}{false}
%%%%%%%%%%%%%%%%%%%%%%%%%%%%%%%%%%%%%%%%%%%%%%%%%%%%%%%%%%%%%%
%%%%%%%%%%%%%%%%%%%%%%%%%%%%%%%%%%%%%%%%%%%%%%%%%%%%%%%%%%%%%%
%%%%%%%%%%%%%%%%%%%%%%%%%%%%%%%%%%%%%%%%%%%%%%%%%%%%%%%%%%%%%M
\ifthenelse{\boolean{sol}}
{
\newenvironment{solution}
{
\begin{examplebox}
\begin{center}
\begin{tabular}{m{0.5cm}|m{15cm}|}\cline{2-2}
%\begin{turn}{90}{Solution}\end{turn}&
\rotatebox{90}{Solution}&
\begin{minipage}[H]{\linewidth}
\vspace{0.2cm}
}
{
\vspace{0.2cm}
\end{minipage}
\\\cline{2-2}
\end{tabular}
\end{center}
\end{examplebox}
}
}
{
\newxcomment[]{solution}
}


\begin{document}

\lhead[ROB3/MAIN3 ]{ROB3/MAIN3 }
\chead[]{}
\rhead[Informatique G\'en\'erale ]{Informatique G\'en\'erale}
 
\ifthenelse{\boolean{sol}}
{
\lfoot[Polytech'Paris-UPMC 2017-2018 (version enseignants)]{Polytech'Paris-UPMC 2017-2018 (version enseignants)}

}
{
\lfoot[Polytech'Paris-UPMC 2017-2018 (version \'etudiants)]{Polytech'Paris-UPMC 2017-2018 (version \'etudiants)}
}

\cfoot[]{}
\rfoot[\thepage]{\thepage}




\pagestyle{fancyplain}
%%%%%%%%%%%%

 %\addcontentsline{toc}{chapter}{S\'erie 1 : Conception d'algorithmes}
 %\chapter{Conception d'algorithmes}
 %\addcontentsline{toc}{chapter}{S\'erie 1 : Conception d'algorithmes}

\chapter*{Travaux Dirigés n°3}

\exercice {} 
Ecrire l'algorithme qui compte le nombre de déplacements effectués par
la tête de lecture/écriture d'une
machine de Turing avant qu'elle ne s'arrête. On considère une machine
de Turing à 5 états de 0 à 4 et un état particulier -1 qui arrête la
machine.
La machine peut lire ou écrire uniquement des 0 ou des 1 sur le ruban.
On modélise la machine de
Turing de la façon suivante
\begin{itemize}
\item Un tableau d'entiers \texttt{ruban} de taille 100 qui représente le ruban
  (normalement infini). \texttt{ruban} sera
  initialisé avec que des zéros.
\item Un entier \texttt{pos} représentant la position du curseur sur le
tableau \texttt{ruban}. \texttt{pos} sera initialisé à $50$.
\item Un entier \texttt{cstate} représentant l'état courant de la
  machine. \texttt{cstate} sera initalisé à 0.
\item Un tableau d'entiers \texttt{write0} (resp. \texttt{write1}) de taille 5
  qui donne la valeur à écrire sur le ruban de la machine dans le cas
  où on lit un 0 (resp. 1). Par exemple, si la machine lit un 0 sur le
  ruban, et que son état courant est 3, alors on effectue l'affectation
  suivante : \texttt{ruban[pos]=write1[3];}. Notez que les tableaux
  \texttt{write0} et \texttt{write1} ne contiennent que des 0 ou des 1.
\item Un tableau d'entiers \texttt{depl0} (resp. \texttt{depl1}) de
  taille 5 qui donne le déplacement de la tête de lecture avec -1 pour
  un déplacement vers la gauche et +1 pour un déplacement vers la
  droite. Par exemple, si la machine lit un 1 sur le ruban et que son
  état courant est 4, alors, la position de la tête de lecture est
  modifiée de la façon suivante : \texttt{pos=pos+depl1[4]}.
\item Un tableau d'entiers  \texttt{nextstate0} (resp.  \texttt{nextstate1)} de taille 5 qui donne
  la valeur du prochain état de la machine dans le cas où on lit un 0
  (resp. 1). Par exemple, si la machine lit un 1 sur le ruban et que
  son état courant est 2, alors, le prochain état sera l'entier
  contenu dans  \texttt{nextstate1[2]} (\texttt{cstate=nexstate1[2]}). Attention, il y a un élément du tableau
  \texttt{nextstate0} ou \texttt{nextstate1} qui est égal à -1 qui
  arrête la machine.\\
\end{itemize}
Pour tester l'algorithme, on prendra la machine de Turing suivante :
\begin{itemize}
\item \texttt{write0=\{1,1,1,1,1\}}
\item \texttt{write1=\{0,1,0,0,0\}}
\item \texttt{depl0=\{1,1,1,1,-1\}}
\item \texttt{depl1=\{-1,-1,-1,-1,1\}}
\item \texttt{nextstate0=\{1,2,3,4,3\}}
\item \texttt{nextstate1=\{-1,0,1,2,4\}}
\end{itemize}

\begin{solution}
%\begin{algorithm}[h!]
%\caption{D\'etermination d'un nombre d'occurences d'un entier dans une saisie}
%\label{algo_cm}
\begin{algorithmic}[1]
\REQUIRE{write0,write1,depl0,depl1,nexstate0,nexstate1 : tableaux d'entiers}
\ENSURE{compte : entier, nombre de déplacement}
\\
{\bf Donn\'ees locales : } pos,cstate : entiers ; ruban[] : tableau d'entiers
\\
\STATE pos $\leftarrow 50$
\STATE cstate $\leftarrow 0$
\STATE compte $\leftarrow 0$
\STATE Initialise ruban avec des zéros
\WHILE{cstate $\neq -1$}
\IF{ruban[pos] $=0$}
\STATE ruban[pos] $\leftarrow$ write0[cstate]
\STATE pos $\leftarrow$ pos + depl0[cstate]
\STATE cstate $\leftarrow$ nextstate0[cstate]
\ELSE
\STATE ruban[pos] $\leftarrow$ write1[cstate]
\STATE pos $\leftarrow$ pos + depl1[cstate]
\STATE cstate $\leftarrow$ nextstate1[cstate]
\ENDIF
\STATE compte=compte+1
\ENDWHILE
\end{algorithmic}
%\end{algorithm}
\end{solution}


\exercice{- factoriel}
Ecrire 2 algorithmes permettant de calculer $n!$. L'un des algorithmes sera itératif et l'autre récursif.

\exercice {- suite de Fibonacci}
La suite de Fibonacci est définie de manière récursive par la relation : $u_n = u_{n-1} + u_{n-2}$.\\
et par les conditions initiales : $u_0=u_1=1$.\\
Construire un algorithme d'abord itératif, puis récursif permettant de calculer le n-ième terme.

\begin{solution}
\begin{verbatim}
#include <stdio.h>

int fibonacci(int);

int main() {
  int result, number;
  printf("Entrer n: ");
  scanf("%d", &number);
  result = fibonacci(number);
  printf("fibonacci(%d) = %d\n", number, result);
  return 0;
}

int fibonacci(int n)
{
  if (n == 0 || n==1)
    return n;
  else return (fibonacci(n-1) + fibonacci(n-2));
}
\end{verbatim}
\end{solution}

\exercice{- Puissance k-ième iterative}
Ecrire un algorithme prenant en entrée deux entiers $n$ et $k$ ($k>0$) qui
renvoie $n^k$. Il est bien entendu interdit d'utiliser une fonction ou un
opérateur calculant directement cette puissance.

\exercice {- Puissance k-ième récursive }
Pour calculer $n^k$ ($n$,$k$, entiers positifs), on peut utiliser le principe de récursion suivant:
\begin{itemize}
\item si $k=0$ alors, le résultat est 1.
\item si $k=2 \times p$ (k est pair), alors on calcule (selon le même principe) $m=n^p$ puis le résultat est $m \times m$.
\item si $k=2 \times p + 1$ (k est impair), alors on calcule (selon le même principe) $m=n^p$ puis le résultat est $n \times m \times m$.
\end{itemize}

\begin{enumerate}
\item Ecrire l'algorithme récursif \verb|puiss| qui prend en entrée deux entiers positifs $n$ et $k$ et renvoie $n^k$ selon le principe décrit ci-dessus.
\item Dans le cas du calcul de $n^{18}$ (n positif), combien d'appels seront fait à l'algorithme \verb|puiss| ?
\item Dans le calcul de $n^k$ avec $k=2^p$, combien d'appels seront fait à l'algotithme \verb|puiss| (la réponse dépend bien sûr de p).
\end{enumerate}

\begin{solution}
\begin{enumerate}
\item Voici la fonction (codé en C)
\begin{verbatim}
int puiss(int n, int k) {
  int m; 
  int rep=1;
  if (k==0) return(1);
  else {
    m=puiss(n,k/2);
    if (k%2==1) rep=n;
    return(rep*m*m);
  }
}
\end{verbatim}

\item  Voici la liste des appels à la fonction \verb|puiss| : \verb|puiss(n,18)| appelle \verb|puiss(n,9)| qui appelle \verb|puiss(n,4)| qui appelle \verb|puiss(n,2)|  qui appelle \verb|puiss(n,1)| qui apelle \verb|puiss(n,0)|. Au total : 6 appels.

\item Pour calculer $n^k$ avec $k=2^p$. La fonction \verb|puiss(n,k)| apelle \verb|puiss(n,k')| avec $k'=2^p/2=2^{p-1}$. Le p se décrémente de 1 à chaque appel jusqu'à appel avec $k=2^0=1$. Il y a donc au total p+1 appels. (Pour p=1, on appelle 2 fois la fonction et ensuite on fait un appel supplémentaire à chaque fois que p est incrémenté de 1).
\end{enumerate}
\end{solution}


\exercice{}

Le problème des tours de Hanoï est un jeu de réflexion imaginé par le mathématicien français Édouard Lucas, et consistant à déplacer des disques de diamètres différents d'une tour de « départ » à une tour d'« arrivée » en passant par une tour « intermédiaire » et ceci en un minimum de coups, tout en respectant les règles suivantes :
\begin{itemize}
    \item on ne peut déplacer plus d'un disque à la fois,
    \item  on ne peut placer un disque que sur un autre disque plus grand que lui ou sur un emplacement vide.
\end{itemize}
On suppose que cette dernière règle est également respectée dans la configuration de départ.
On suppose que les trois positions des trois tours sont numérotées 1, 2 et 3.

Ecrire un algorithme qui prend en entrée $n$ le nombre de disque sur
la tour, $dep$ la position de la tour de départ, $med$ la position de
la tour intermédiaire, et $fin$ la position finale. Cet algorithme ne renvoie rien mais  affiche les déplacement successifs sous forme d'un couple d'entier $(a,b)$ qui signifie qu'on déplace le disque au dessus de la tour $a$ sur la tour $b$.

Exemple : les mouvements nécessaires pour déplacer 3 disques de $1$ vers $3$ en passant par $2$ :
\begin{itemize}
\item (1,3)
\item (1,2)
\item (3,2)
\item (1,3)
\item (2,1)
\item (2,3)
\item (1,3)
\end{itemize}

\begin{solution}
\begin{verbatim}
void deplace(int n, int dep,int med,int fin)
{
if (n==1) printf("(%d,%d)\n",dep,fin);
if (n>1) {
	deplace (n-1,dep,fin,med);
	printf("(%d,%d)\n",dep,fin);
	deplace (n-1,med,dep,fin);
	 }
}

\end{verbatim}

\end{solution}



%%%%%%%%%%%%%%%%%%%%%%%%%%%%%%%%%%%%%%%%%%%%%%%%%%%%%%%%%%%%%%%%%%%%%%%%%


\end{document}
