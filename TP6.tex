\documentclass[french,12pt,a4paper,twoside,openright,titlepage]{report}
\usepackage{algorithm}
\usepackage{algorithmic}
\usepackage[latin1]{inputenc}
\usepackage{fancyhdr}
\usepackage{amsmath}
\usepackage{amssymb}
\usepackage{babel}
\usepackage{amsfonts}
\usepackage{graphicx}
%\usepackage{here}
\usepackage{ifthen}
\usepackage{xcomment}
\usepackage{array}
\usepackage{rotating}

\textwidth17cm
\textheight25cm
%\oddsidemargin-1cm
%\parskip0.2pt
\voffset-2cm
\hoffset-1cm
\headsep0.5cm
\topmargin0.5cm
\oddsidemargin0.5cm
\evensidemargin0.5cm

\makeatletter
\@addtoreset{section}{part}
\makeatother
\newlength{\mylength}
\newcounter{exo}\stepcounter{exo}
\newcommand\exercice{
\subsubsection*{Exercice \theexo\stepcounter{exo}}
%{\bf Exercice \theexo\stepcounter{exo}\\}
}



\RequirePackage{layout,color}

\definecolor{gray50}{gray}{.5}
\definecolor{gray40}{gray}{.6}
\definecolor{gray30}{gray}{.7}
\definecolor{gray20}{gray}{.8}
\definecolor{gray10}{gray}{.9}
\definecolor{gray05}{gray}{.95}

\newsavebox{\laboiboite}
\newlength{\longdelaboiboite}
\newlength{\deptdelaboiboite}
\newlength{\hautdelaboiboite}
\newcolumntype{R}{>{\begin{lrbox}{\laboiboite} 
$\displaystyle }r<{$\end{lrbox}%
    \settowidth{\longdelaboiboite}{\usebox{\laboiboite}}%
    \settoheight{\hautdelaboiboite}{\usebox{\laboiboite}}%
    \settodepth{\deptdelaboiboite}{\usebox{\laboiboite}}%
    \addtolength{\hautdelaboiboite}{\deptdelaboiboite}%
    \addtolength{\hautdelaboiboite}{2ex}%
    \parbox[c][\hautdelaboiboite][c]{0cm}{} \makebox[\width]
{\usebox{\laboiboite}}}}
\newcolumntype{C}{>{\begin{lrbox}{\laboiboite} 
$\displaystyle }c<{$\end{lrbox}%
    \settowidth{\longdelaboiboite}{\usebox{\laboiboite}}%
    \settoheight{\hautdelaboiboite}{\usebox{\laboiboite}}%
    \settodepth{\deptdelaboiboite}{\usebox{\laboiboite}}%
    \addtolength{\hautdelaboiboite}{\deptdelaboiboite}%
    \addtolength{\hautdelaboiboite}{2ex}%
    \parbox[c][\hautdelaboiboite][c]{0cm}{} \makebox[\width]
{\usebox{\laboiboite}}}}
\newcolumntype{L}{>{\begin{lrbox}{\laboiboite} 
$\displaystyle }l<{$\end{lrbox}%
    \settowidth{\longdelaboiboite}{\usebox{\laboiboite}}%
    \settoheight{\hautdelaboiboite}{\usebox{\laboiboite}}%
    \settodepth{\deptdelaboiboite}{\usebox{\laboiboite}}%
    \addtolength{\hautdelaboiboite}{\deptdelaboiboite}%
    \addtolength{\hautdelaboiboite}{2ex}%
    \parbox[c][\hautdelaboiboite][c]{0cm}{} \makebox[\width]
{\usebox{\laboiboite}}}}

\newcolumntype{S}{>{\begin{lrbox}{\laboiboite}}r<{\end{lrbox}%
    \mbox{\usebox{\laboiboite}}}}
\newcolumntype{D}{>{\begin{lrbox}{\laboiboite}}c<{\end{lrbox}%
    \mbox{\usebox{\laboiboite}}}}
\newcolumntype{M}{>{\begin{lrbox}{\laboiboite}}l<{\end{lrbox}%
    \mbox{\usebox{\laboiboite}}}}
\newlength\Linewidth
\def\findlength{\setlength\Linewidth\linewidth
\addtolength\Linewidth{-4\fboxrule}
\addtolength\Linewidth{-3\fboxsep}
}
\newenvironment{examplebox}{\par\begingroup%
   \setlength{\fboxsep}{5pt}\findlength%
   \setbox0=\vbox\bgroup\noindent%
   \hsize=\Linewidth%
   \begin{minipage}{\Linewidth}\small}%
    {\end{minipage}\egroup%
    \vspace{6pt}%
    \noindent\textcolor{gray20}{\fboxsep2.5pt\fbox%
     {\fboxsep5pt\colorbox{gray05}{\normalcolor\box0}}}%
    \endgroup\par\addvspace{6pt minus 3pt}\noindent%
    \normalcolor\ignorespacesafterend}
\let\Examplebox\examplebox
\let\endExamplebox\endexamplebox


%%%%%%%%%%%%%%%%%%%%%%%%%%%%%%%%%%%%%%%%%%%%%%%%%%%%%%%%%%%%%%
%%%%%%%%%%%%%%%%%%%%%%%%%%%%%%%%%%%%%%%%%%%%%%%%%%%%%%%%%%%%%%
%%%%%%%%%%%%%%%%%%%%%%%%%%%%%%%%%%%%%%%%%%%%%%%%%%%%%%%%%%%%%%
\newboolean{sol}
%\setboolean{sol}{true}
\setboolean{sol}{false}
%%%%%%%%%%%%%%%%%%%%%%%%%%%%%%%%%%%%%%%%%%%%%%%%%%%%%%%%%%%%%%
%%%%%%%%%%%%%%%%%%%%%%%%%%%%%%%%%%%%%%%%%%%%%%%%%%%%%%%%%%%%%%
%%%%%%%%%%%%%%%%%%%%%%%%%%%%%%%%%%%%%%%%%%%%%%%%%%%%%%%%%%%%%M
\ifthenelse{\boolean{sol}}
{
\newenvironment{solution}
{
\begin{examplebox}
\begin{center}
\begin{tabular}{m{0.5cm}|m{15cm}|}\cline{2-2}
%\begin{turn}{90}{Solution}\end{turn}&
\rotatebox{90}{Solution}&
\begin{minipage}[H]{\linewidth}
\vspace{0.2cm}
}
{
\vspace{0.2cm}
\end{minipage}
\\\cline{2-2}
\end{tabular}
\end{center}
\end{examplebox}
}
}
{
\newxcomment[]{solution}
}


\begin{document}

\lhead[ROB3/MAIN3]{ROB3/MAIN3}
\chead[]{}
\rhead[Informatique G\'en\'erale ]{Informatique G\'en\'erale}
 
\ifthenelse{\boolean{sol}}
{
\lfoot[Polytech'Paris-UPMC 2017-2018 (version enseignants)]{Polytech'Paris-UPMC 2017-2018 (version enseignants)}

}
{
\lfoot[Polytech'Paris-UPMC 2017-2018 (version \'etudiants)]{Polytech'Paris-UPMC 2017-2018 (version \'etudiants)}
}

\cfoot[]{}
\rfoot[\thepage]{\thepage}




\pagestyle{fancyplain}
%%%%%%%%%%%%

 %\addcontentsline{toc}{chapter}{S\'erie 1 : Conception d'algorithmes}
 %\chapter{Conception d'algorithmes}
 %\addcontentsline{toc}{chapter}{S\'erie 1 : Conception d'algorithmes}

\chapter*{Travaux Pratiques th�me 6 : Les listes cha�n�es}

\begin{solution}
La solution de ce TP n'est pas encore disponible.
\end{solution}

\exercice
D�finir un type de donn�e liste dont chaque maillon est compos�  d'un nombre flottant $c$ en double pr�cision et d'un entier $n$.

\exercice
Ecrire une fonction permettant d'ajouter un �l�ment dans une liste ordonn�e. La t�te de liste correspondant au maillon dont l'entier $n$ est le plus grand. Le dernier maillon aura l'entier $n$ le plus petit. On ne s'occupe pas pour l'instant des cas d'�galit�.

\exercice
Ecrire et tester les fonctions permettant d'afficher l'ensemble d'une liste et de supprimer le premier maillon de la liste.

\exercice
Modifier la fonction d'ajout pour g�rer les cas d'�galit�. Si on cherche ins�rer dans la liste un �l�ment $(c_1,n)$ et qu'il existe d�j� un maillon $(c_2,n)$ dans la liste, on modifie le maillon existant pour lui donner la valeur $(c_1+c_2,n)$.

\exercice
Ecrire une fonction permettant d'initialiser la liste. La fonction demande � l'utilisateur de rentrer des valeurs et s'arr�te d�s que l'utilisateur rentre une valeur de $n$ n�gative.

\exercice
La liste que nous avons d�finie pr�c�demment peut permettre de repr�senter les polyn�mes $\sum_{n=0}^pc_n.x^n$ � coefficients r�els. Le champs $c$ du maillon repr�sentant le coefficient et le champs $n$ le degr� du terme consid�r�. Nous allons maintenant programmer certaines fonctions usuelles des polyn�mes en se servant de cette structure de liste.  
\begin{enumerate}
\item La fonction qui renvoie la valeur du polyn�me pour un $x$ particulier.
\item La fonction qui fait la somme de deux polyn�mes.
\item La fonction qui multiplie un polyn�me par un r�el.
\item La fonction qui multiplie un polyn�me par un monome de la forme $c.x^n$.
\end{enumerate}


\end{document}
