\documentclass[french,12pt,a4paper,twoside,openright,titlepage]{report}
\usepackage{algorithm}
\usepackage{algorithmic}
\usepackage[latin1]{inputenc}
\usepackage{fancyhdr}
\usepackage{amsmath}
\usepackage{amssymb}
\usepackage[french]{babel}
\usepackage{amsfonts}
\usepackage{graphicx}
%\usepackage{here}
\usepackage{ifthen}
\usepackage{xcomment}
\usepackage{array}
\usepackage{rotating}

\textwidth17cm
\textheight25cm
%\oddsidemargin-1cm
%\parskip0.2pt
\voffset-2cm
\hoffset-1cm
\headsep0.5cm
\topmargin0.5cm
\oddsidemargin0.5cm
\evensidemargin0.5cm

\makeatletter
\@addtoreset{section}{part}
\makeatother
\newlength{\mylength}
\newcounter{exo}\stepcounter{exo}
\newcommand\exercice[1]{
\subsubsection*{Exercice \theexo\stepcounter{exo} {#1}}
%{\bf Exercice \theexo\stepcounter{exo}\\}
}


\RequirePackage{layout,color}

\definecolor{gray50}{gray}{.5}
\definecolor{gray40}{gray}{.6}
\definecolor{gray30}{gray}{.7}
\definecolor{gray20}{gray}{.8}
\definecolor{gray10}{gray}{.9}
\definecolor{gray05}{gray}{.95}

\newsavebox{\laboiboite}
\newlength{\longdelaboiboite}
\newlength{\deptdelaboiboite}
\newlength{\hautdelaboiboite}
\newcolumntype{R}{>{\begin{lrbox}{\laboiboite} 
$\displaystyle }r<{$\end{lrbox}%
    \settowidth{\longdelaboiboite}{\usebox{\laboiboite}}%
    \settoheight{\hautdelaboiboite}{\usebox{\laboiboite}}%
    \settodepth{\deptdelaboiboite}{\usebox{\laboiboite}}%
    \addtolength{\hautdelaboiboite}{\deptdelaboiboite}%
    \addtolength{\hautdelaboiboite}{2ex}%
    \parbox[c][\hautdelaboiboite][c]{0cm}{} \makebox[\width]
{\usebox{\laboiboite}}}}
\newcolumntype{C}{>{\begin{lrbox}{\laboiboite} 
$\displaystyle }c<{$\end{lrbox}%
    \settowidth{\longdelaboiboite}{\usebox{\laboiboite}}%
    \settoheight{\hautdelaboiboite}{\usebox{\laboiboite}}%
    \settodepth{\deptdelaboiboite}{\usebox{\laboiboite}}%
    \addtolength{\hautdelaboiboite}{\deptdelaboiboite}%
    \addtolength{\hautdelaboiboite}{2ex}%
    \parbox[c][\hautdelaboiboite][c]{0cm}{} \makebox[\width]
{\usebox{\laboiboite}}}}
\newcolumntype{L}{>{\begin{lrbox}{\laboiboite} 
$\displaystyle }l<{$\end{lrbox}%
    \settowidth{\longdelaboiboite}{\usebox{\laboiboite}}%
    \settoheight{\hautdelaboiboite}{\usebox{\laboiboite}}%
    \settodepth{\deptdelaboiboite}{\usebox{\laboiboite}}%
    \addtolength{\hautdelaboiboite}{\deptdelaboiboite}%
    \addtolength{\hautdelaboiboite}{2ex}%
    \parbox[c][\hautdelaboiboite][c]{0cm}{} \makebox[\width]
{\usebox{\laboiboite}}}}

\newcolumntype{S}{>{\begin{lrbox}{\laboiboite}}r<{\end{lrbox}%
    \mbox{\usebox{\laboiboite}}}}
\newcolumntype{D}{>{\begin{lrbox}{\laboiboite}}c<{\end{lrbox}%
    \mbox{\usebox{\laboiboite}}}}
\newcolumntype{M}{>{\begin{lrbox}{\laboiboite}}l<{\end{lrbox}%
    \mbox{\usebox{\laboiboite}}}}
\newlength\Linewidth
\def\findlength{\setlength\Linewidth\linewidth
\addtolength\Linewidth{-4\fboxrule}
\addtolength\Linewidth{-3\fboxsep}
}
\newenvironment{examplebox}{\par\begingroup%
   \setlength{\fboxsep}{5pt}\findlength%
   \setbox0=\vbox\bgroup\noindent%
   \hsize=\Linewidth%
   \begin{minipage}{\Linewidth}\small}%
    {\end{minipage}\egroup%
    \vspace{6pt}%
    \noindent\textcolor{gray20}{\fboxsep2.5pt\fbox%
     {\fboxsep5pt\colorbox{gray05}{\normalcolor\box0}}}%
    \endgroup\par\addvspace{6pt minus 3pt}\noindent%
    \normalcolor\ignorespacesafterend}
\let\Examplebox\examplebox
\let\endExamplebox\endexamplebox


%%%%%%%%%%%%%%%%%%%%%%%%%%%%%%%%%%%%%%%%%%%%%%%%%%%%%%%%%%%%%%
%%%%%%%%%%%%%%%%%%%%%%%%%%%%%%%%%%%%%%%%%%%%%%%%%%%%%%%%%%%%%%
%%%%%%%%%%%%%%%%%%%%%%%%%%%%%%%%%%%%%%%%%%%%%%%%%%%%%%%%%%%%%%
\newboolean{sol}
%\setboolean{sol}{true}
\setboolean{sol}{false}
%%%%%%%%%%%%%%%%%%%%%%%%%%%%%%%%%%%%%%%%%%%%%%%%%%%%%%%%%%%%%%
%%%%%%%%%%%%%%%%%%%%%%%%%%%%%%%%%%%%%%%%%%%%%%%%%%%%%%%%%%%%%%
%%%%%%%%%%%%%%%%%%%%%%%%%%%%%%%%%%%%%%%%%%%%%%%%%%%%%%%%%%%%%M
\ifthenelse{\boolean{sol}}
{
\newenvironment{solution}
{
\begin{examplebox}
\begin{center}
\begin{tabular}{m{0.5cm}|m{15cm}|}\cline{2-2}
%\begin{turn}{90}{Solution}\end{turn}&
\rotatebox{90}{Solution}&
\begin{minipage}[H]{\linewidth}
\vspace{0.2cm}
}
{
\vspace{0.2cm}
\end{minipage}
\\\cline{2-2}
\end{tabular}
\end{center}
\end{examplebox}
}
}
{
\newxcomment[]{solution}
}


\begin{document}

\lhead[ROB3 ]{ROB3 }
\chead[]{}
\rhead[Informatique G\'en\'erale ]{Informatique G\'en\'erale}
 
\ifthenelse{\boolean{sol}}
{
\lfoot[Polytech'Paris-UPMC 2015-2016 (version enseignants)]{Polytech'Paris-UPMC 2015-2016 (version enseignants)}

}
{
\lfoot[Polytech'Paris-UPMC 2015-2016 (version \'etudiants)]{Polytech'Paris-UPMC 2015-2016 (version \'etudiants)}
}

\cfoot[]{}
\rfoot[\thepage]{\thepage}




\pagestyle{fancyplain}
%%%%%%%%%%%%

 %\addcontentsline{toc}{chapter}{S\'erie 1 : Conception d'algorithmes}
 %\chapter{Conception d'algorithmes}
 %\addcontentsline{toc}{chapter}{S\'erie 1 : Conception d'algorithmes}

\chapter*{Travaux Dirig�s n�2}

\exercice{- Minimum}
Ecrire l'algorithme qui renvoie l'indice du plus petit �lement
d'un tableau d'entiers.

\exercice{- Nombre d'occurences}
Ecrire un algorithme qui prend en entr�e un tableau d'entiers, et un entier
$n$ et qui renvoie le nombre de fois o� l'entier $n$ appara�t dans le tableau.

\exercice{- Moyenne et �cart-type}
Ecrire un algorithme qui calcule la moyenne et l'�cart-type des �l�ments d'un tableau de r�els.
Les formules pour la moyenne $m$ et l'�cart-type $\sigma$ d'un tableau $t$ de taille $n$ sont :
$$
m=\frac{1}{n}\sum_{i=0}^{n-1}t[i]
$$
$$
\sigma=\sqrt{\frac{1}{n-1}\sum_{i=0}^{n-1}(t[i]-m)^2}
$$



\exercice{- Palindrome}
Ecrire l'algorithme qui prend un tableau d'entiers de taille N en entr�e, et v�rifie que
ce tableau est un palindrome, c'est � dire qu'il se lit de la m�me fa�on dans un sens ou dans l'autre.
Exemple : les tableaux \verb|{1,7,2,7,1}| ou \verb|{3,4,4,3}| sont des palindromes.



\exercice{- Remplacement des doublons}
Soit un tableau \verb|T| contenant des entiers strictement positifs.
Ecrire l'algorithme qui remplace tous les doublons par des 0.
Exemple : le tableau \verb|{1,3,3,1,2,5,1}| deviendra \verb|{1,3,0,0,2,5,0}|.

\exercice {- Tri par s�lection} 
\'Ecrire l'algorithme de tri par s�lection qui se base sur le principe suivant. 
On s�lectionne tout d'abord l'�l�ment le plus petit du tableau, c.� d. on trouve l'entier 
$p$ tel que $\forall 1 \leq i \leq n, t[i] \geq t[p]$. Une fois cet emplacement trouv�, on �change les �lements 
$t[1]$ et $t[p]$. Puis on recommence ces op�rations pour le reste du tableau (c.� d. les �l�ments 
compris entre les indices $2$ et $n$. On recherche alors le plus petit �l�ment de cette nouvelle suite 
de nombre et on �change avec $t[2]$. Et ainsi de suite \ldots jusqu'au moment o� on a plac� tous les 
�lements du tableau.
 

\begin{solution}
%\begin{algorithm}[h!]
%\caption{Calculatrice simple}
%\label{algo_cm}
\begin{algorithmic}[1]
\REQUIRE{$T$ : Tableau d'entiers \`a trier ; $n$ : taille du tableau}
\ENSURE{$T$ : Tableau tri \'e}
\\
{\bf Donn\'ees locales : } $i$,$j$,$imin$,$temp$ entiers.
\\
\FOR{$i$ allant de $2$ \`a $n-1$}
\STATE $imin \leftarrow i$
\FOR {$j$ allant de $i+1$ \`a $n$}
\IF{$T[imin]>T[j]$}
  \STATE $imin \leftarrow j$
\ENDIF
\STATE $temp \leftarrow T[i]$
\STATE $T[i] \leftarrow T[imin]$
\STATE $T[imin] \leftarrow temp$
\ENDFOR
\ENDFOR
 \end{algorithmic}
%\end{algorithm}
\end{solution}
%%%%%%%%%%%%%%%%%%%%%%%%%%%%%%%%%%%%%%%%%%%%%%%%%%%%%%%%%%%%%%%%%%%%%%%%%


%%%%%%%%%%%%%%%%%%%%%%%%%%%%%%%%%%%%%%%%%%%%%%%%%%%%%%%%%%%%%%%%%%%%%%%%%


\end{document}
