\documentclass[french,12pt,a4paper,twoside,openright,titlepage]{report}
\usepackage{algorithm}
\usepackage{algorithmic}
\usepackage[latin1]{inputenc}
\usepackage{fancyhdr}
\usepackage{amsmath}
\usepackage{amssymb}
\usepackage{babel}
\usepackage{amsfonts}
\usepackage{graphicx}
%\usepackage{here}
\usepackage{ifthen}
\usepackage{xcomment}
\usepackage{array}
\usepackage{rotating}

\textwidth17cm
\textheight25cm
%\oddsidemargin-1cm
%\parskip0.2pt
\voffset-2cm
\hoffset-1cm
\headsep0.5cm

\topmargin0.5cm
\oddsidemargin0.5cm
\evensidemargin0.5cm

\makeatletter
\@addtoreset{section}{part}
\makeatother
\newlength{\mylength}
\newcounter{exo}\stepcounter{exo}
\newcommand\exercice{
\subsubsection*{Exercice \theexo\stepcounter{exo}}
%{\bf Exercice \theexo\stepcounter{exo}\\}
}



\RequirePackage{layout,color}

\definecolor{gray50}{gray}{.5}
\definecolor{gray40}{gray}{.6}
\definecolor{gray30}{gray}{.7}
\definecolor{gray20}{gray}{.8}
\definecolor{gray10}{gray}{.9}
\definecolor{gray05}{gray}{.95}

\newsavebox{\laboiboite}
\newlength{\longdelaboiboite}
\newlength{\deptdelaboiboite}
\newlength{\hautdelaboiboite}
\newcolumntype{R}{>{\begin{lrbox}{\laboiboite} 
$\displaystyle }r<{$\end{lrbox}%
    \settowidth{\longdelaboiboite}{\usebox{\laboiboite}}%
    \settoheight{\hautdelaboiboite}{\usebox{\laboiboite}}%
    \settodepth{\deptdelaboiboite}{\usebox{\laboiboite}}%
    \addtolength{\hautdelaboiboite}{\deptdelaboiboite}%
    \addtolength{\hautdelaboiboite}{2ex}%
    \parbox[c][\hautdelaboiboite][c]{0cm}{} \makebox[\width]
{\usebox{\laboiboite}}}}
\newcolumntype{C}{>{\begin{lrbox}{\laboiboite} 
$\displaystyle }c<{$\end{lrbox}%
    \settowidth{\longdelaboiboite}{\usebox{\laboiboite}}%
    \settoheight{\hautdelaboiboite}{\usebox{\laboiboite}}%
    \settodepth{\deptdelaboiboite}{\usebox{\laboiboite}}%
    \addtolength{\hautdelaboiboite}{\deptdelaboiboite}%
    \addtolength{\hautdelaboiboite}{2ex}%
    \parbox[c][\hautdelaboiboite][c]{0cm}{} \makebox[\width]
{\usebox{\laboiboite}}}}
\newcolumntype{L}{>{\begin{lrbox}{\laboiboite} 
$\displaystyle }l<{$\end{lrbox}%
    \settowidth{\longdelaboiboite}{\usebox{\laboiboite}}%
    \settoheight{\hautdelaboiboite}{\usebox{\laboiboite}}%
    \settodepth{\deptdelaboiboite}{\usebox{\laboiboite}}%
    \addtolength{\hautdelaboiboite}{\deptdelaboiboite}%
    \addtolength{\hautdelaboiboite}{2ex}%
    \parbox[c][\hautdelaboiboite][c]{0cm}{} \makebox[\width]
{\usebox{\laboiboite}}}}

\newcolumntype{S}{>{\begin{lrbox}{\laboiboite}}r<{\end{lrbox}%
    \mbox{\usebox{\laboiboite}}}}
\newcolumntype{D}{>{\begin{lrbox}{\laboiboite}}c<{\end{lrbox}%
    \mbox{\usebox{\laboiboite}}}}
\newcolumntype{M}{>{\begin{lrbox}{\laboiboite}}l<{\end{lrbox}%
    \mbox{\usebox{\laboiboite}}}}
\newlength\Linewidth
\def\findlength{\setlength\Linewidth\linewidth
\addtolength\Linewidth{-4\fboxrule}
\addtolength\Linewidth{-3\fboxsep}
}
\newenvironment{examplebox}{\par\begingroup%
   \setlength{\fboxsep}{5pt}\findlength%
   \setbox0=\vbox\bgroup\noindent%
   \hsize=\Linewidth%
   \begin{minipage}{\Linewidth}\small}%
    {\end{minipage}\egroup%
    \vspace{6pt}%
    \noindent\textcolor{gray20}{\fboxsep2.5pt\fbox%
     {\fboxsep5pt\colorbox{gray05}{\normalcolor\box0}}}%
    \endgroup\par\addvspace{6pt minus 3pt}\noindent%
    \normalcolor\ignorespacesafterend}
\let\Examplebox\examplebox
\let\endExamplebox\endexamplebox


%%%%%%%%%%%%%%%%%%%%%%%%%%%%%%%%%%%%%%%%%%%%%%%%%%%%%%%%%%%%%%
%%%%%%%%%%%%%%%%%%%%%%%%%%%%%%%%%%%%%%%%%%%%%%%%%%%%%%%%%%%%%%
%%%%%%%%%%%%%%%%%%%%%%%%%%%%%%%%%%%%%%%%%%%%%%%%%%%%%%%%%%%%%%
\newboolean{sol}
\setboolean{sol}{true}
%\setboolean{sol}{false}
%%%%%%%%%%%%%%%%%%%%%%%%%%%%%%%%%%%%%%%%%%%%%%%%%%%%%%%%%%%%%%
%%%%%%%%%%%%%%%%%%%%%%%%%%%%%%%%%%%%%%%%%%%%%%%%%%%%%%%%%%%%%%
%%%%%%%%%%%%%%%%%%%%%%%%%%%%%%%%%%%%%%%%%%%%%%%%%%%%%%%%%%%%%M
\ifthenelse{\boolean{sol}}
{
\newenvironment{solution}
{
\begin{examplebox}
\begin{center}
\begin{tabular}{m{0.5cm}|m{15cm}|}\cline{2-2}
%\begin{turn}{90}{Solution}\end{turn}&
\rotatebox{90}{Solution}&
\begin{minipage}[H]{\linewidth}
\vspace{0.2cm}
}
{
\vspace{0.2cm}
\end{minipage}
\\\cline{2-2}
\end{tabular}
\end{center}
\end{examplebox}
}
}
{
\newxcomment[]{solution}
}


\begin{document}

\lhead[ROB3/MAIN3]{ROB3/MAIN3}
\chead[]{}
\rhead[Informatique G\'en\'erale ]{Informatique G\'en\'erale}
 
\ifthenelse{\boolean{sol}}
{
\lfoot[Polytech'Paris-UPMC 2017-2018 (version enseignants)]{Polytech'Paris-UPMC 2017-2018 (version enseignants)}

}
{
\lfoot[Polytech'Paris-UPMC 2017-2018 (version \'etudiants)]{Polytech'Paris-UPMC 2017-2018 (version \'etudiants)}
}

\cfoot[]{}
\rfoot[\thepage]{\thepage}

\pagestyle{fancyplain} 
%%%%%%%%%%%%

 %\addcontentsline{toc}{chapter}{S\'erie 1 : Conception d'algorithmes}
 %\chapter{Conception d'algorithmes}
 %\addcontentsline{toc}{chapter}{S\'erie 1 : Conception d'algorithmes}

\chapter*{Travaux Pratiques th�me 4 : Les pointeurs et les tableaux dynamiques}

\section*{Pointeurs}

\exercice

Ecrire une fonction incremente qui prend 3 param�tres en entr�e (\textbf{a}, \textbf{b} et \textbf{i}), et qui ajoute � 
\textbf{a} et \textbf{b} la valeur \textbf{i}. Cette fonction retourne la valeur 1 si \textbf{i} vaut 0, 0 sinon. 

\begin{solution}
\begin{verbatim}
#include <stdio.h> 

int incremente (int *a, int *b, int i) { 
  *a += i; 
  *b += i; 
  if (i) 
    return 0; 
  return 1; 
} 

int main (void){ 
  int p1, p2, i, retour; 
  printf("valeur du premier parametre ? "); 
  scanf("%d", &p1); 
  printf("valeur du second parametre ? "); 
  scanf("%d", &p2); 
  printf("increment ? "); 
  scanf("%d", &i); 
  printf("Avant appel : param1 = %d, param2 = %d, increment = %d\n", p1, p2, i); 
  retour = incremente(&p1, &p2, i); 
  printf("Apres appel : param1 = %d, param2 = %d, increment = %d\n", p1, p2, i); 
  printf("retour de l�appel = %d\n", retour); 
  return 0; 
} 
\end{verbatim}
\end{solution}



\exercice

Ecrire le code C permettant de cr�er un tableau \textbf{tab} de 17 entiers en utilisant un pointeur et la fonction malloc. 
Lib�rer ensuite la m�moire allou�e au tableau avec la fonction free.



\begin{solution}
\begin{verbatim}
#include <stdio.h> 
#include <stdlib.h> 

int main(void) { 
  int * tab; 
  
  tab = (int *) malloc(17 * sizeof(int)); 

  free(tab); 
  
  return 0; 
} 


\end{verbatim}
\end{solution}



\exercice

Ecrire le code C permettant de cr�er un tableau \textbf{tab} � deux dimensions de 15 lignes et 17 colonnes entiers en utilisant un pointeur et la fonction malloc.
Lib�rer ensuite la m�moire allou�e au tableau avec la fonction free.



\begin{solution}
\begin{verbatim}
#include <stdio.h> 
#include <stdlib.h> 

int main(void) 
{ 
  int ** tab; 
  int i; 
  
  tab = (int **) malloc(15 * sizeof(int *)); 
  for (i = 0; i < 15; i++) { 
    tab[i] = (int *) malloc(17 * sizeof(int)); 
  }
 
  for (i = 0; i < 15; i++) { 
    free(tab[i]); 
  } 
  free(tab); 
  
  return 0; 
} 

\end{verbatim}
\end{solution}


\section*{Tableaux et Matrices}


 \exercice
\begin{itemize}
\item Ecrire une fonction qui demande � l'utilisateur le nombre de lignes et de 
colonnes d'une matrices d'entiers, puis remplit la matrice en demandant les 
valeurs � l'utilisateur.

\item Ecrire une fonction {\tt affiche\_mat(m,l,c)} qui permet d'afficher une 
matrice m de {\tt l} lignes et {\tt c} colonnes.
\end{itemize}

\begin{solution}
\begin{verbatim}
int **saisir()
{
	int i, j, lig, col, **m;

	printf("Entrer le nb de lignes et de colonnes\n");
   scanf("%d %d",&lig,&col);
	for(i=0;i<lig;i++)
		for(j=0;j<col;j++)
		{
			printf("Entrer l'element M(%d,%d) : ",i,j);
			scanf("%d",&m[i][j]);
		}
   return m;
}

afficher(int **mat, int col, int lig)
{
	int i, j;

	for(i=0;i<lig;i++)
	{
		for(j=0;j<col;j++)
			printf("%d ",mat[i][j]);
		printf("\n");
	}
}
\end{verbatim}
\end{solution}

\exercice

 Ecrire une fonction permettant de multiplier deux matrices. On n'oubliera 
pas de
traiter les cas d'erreur.


\begin{solution}
{\scriptsize
\begin{verbatim}



multiplication(int **mat1, int col1, int lig1, int **mat2, int col2, int lig2,
int **mat_res){
   int i, j, k, s;

   if((col1 == lig2) && (lig1 == col2))
      for(i=0;i<lig1;i++)
         for(j=0;j<col2;j++){
            s = 0;
            for(k=0;k<col1;k++)
               s = s + mat1[i][k] * mat2[k][j];
            mat_res[i][j] = s;
         }
   else{
      printf("erreur les deux matrices n'ont pas les bonnes tailles\n");
      exit(0);
   }
}

\end{verbatim}
}
\end{solution}

\newpage
\exercice
Le but de cet exercice est de calculer le d�terminant d'une matrice carr�e $n \times n$.
\begin{enumerate}
\item Programmer une fonction qui prends une matrice carr�e ($n>1$) en entr�e et deux entiers $i$ et $j$ et qui renvoie la m�me matrice priv�e de la ligne $i$ et de la colonne $j$.
\item Programmer la fonction \texttt{det} en utilisant la formule suivante : 
\begin{equation}
det(M)=\sum_{i=1}^{n}(m_{ij} \times Cof_{i,j})
\end{equation}
dans laquelle $m_{ij}$ est le terme de ligne $i$ et de colonne $j$ de la matrice $M$, et $Cof_{i,j}$ est �gal � $-1^{i+j}$ multipli� par le d�terminant de la matrice $M$ priv�e de la ligne $i$ et de la colonne $j$.

On notera que le d�terminant d'une matrice M de taille $1 \times 1$ est �gale � la valeur contenue dans la matrice.
\end{enumerate}

\begin{solution}

Le code de cette solution est tr�s moche (d�sol�), mais il fonctionne et contient les fonctionnalit� utiles. la fonction \texttt{m1} calcule $-1^{i+j}$.
Pour info, le d�terminant de la matrice �crite dans la solution est 152.

\begin{verbatim}
#include <stdlib.h>
#include <stdio.h>
void affiche(float **M,int n)
{
  int i,j;
  for (i=0;i<n;i++) {
    for(j=0;j<n;j++) {
      printf("%f\t",M[i][j]);
    }
    printf("\n");
  }
}
float ** init (int n)
{
  int i; 
  float **M;
  M=(float **)malloc(n*sizeof(float));
  for (i=0;i<n;i++)
    {
      M[i]=(float *)malloc(n*sizeof(float));
    }
  return M;
}
int m1 (int i,int j)
{
  if ((i+j) % 2 == 0) return (1);
  else return (-1);
}

float ** extract (float **M,int n,int i,int j)
{
  int ii,jj,l=-1,c=-1;
  int ii,jj;
  float **Me;
  Me=init(n-1);
  for(ii=0;ii<n;ii++) {
    if (ii!=i) {
      l++;c=-1;
      for (jj=0;jj<n;jj++) {
	if (jj!=j)
	  {
	    c++;
	    Me[l][c]=M[ii][jj];
	  }
      }
    }
  }
  return(Me);
}
\end{verbatim}
\end{solution}

%\newpage
\begin{solution}
(suite)
\begin{verbatim}
float det(float **M,int n)
{
  int j=0;
  int i;
  float D=0;
  
  if (n==1)
    {
      return M[0][0];
    }
  else
    {
      for (i=0;i<n;i++)
	{
	  printf("\nn=%d,D=%f\n,m1=%d",n,D,m1(i,j));
	  D += M[i][j]*m1(i,j) * det(extract(M,n,i,j),n-1);
       }
      return(D);
    }
}

int main()
{

  float **M;
  //float **Me;
  float Ms[4][4]={{1,2,3,4},{10,20,11,21},{21,24,27,29},{12,14,15,16}};
  int i,j;
  M=init(4);
 for(i=0;i<4;i++) {
   for(j=0;j<4;j++) {
     M[i][j]=Ms[i][j];
   }
 }
  
 affiche(M,4);
 printf("\n-------------------------\n");
 //Me=extract(M,4,0,0);
 printf("\nD�terminant=%f\n",det(M,4));

  

  return(0);
}

\end{verbatim}
\end{solution}
\end{document}
