\exercice {} 
Ecrire l'algorithme qui compte le nombre de déplacements effectués par
la tête de lecture/écriture d'une
machine de Turing avant qu'elle ne s'arrête. On considère une machine
de Turing à 5 états de 0 à 4 et un état particulier -1 qui arrête la
machine.
La machine peut lire ou écrire uniquement des 0 ou des 1 sur le ruban.
On modélise la machine de
Turing de la façon suivante
\begin{itemize}
\item Un tableau d'entiers \texttt{ruban} de taille 100 qui représente le ruban
  (normalement infini). \texttt{ruban} sera
  initialisé avec que des zéros.
\item Un entier \texttt{pos} représentant la position du curseur sur le
tableau \texttt{ruban}. \texttt{pos} sera initialisé à $50$.
\item Un entier \texttt{cstate} représentant l'état courant de la
  machine. \texttt{cstate} sera initalisé à 0.
\item Un tableau d'entiers \texttt{write0} (resp. \texttt{write1}) de taille 5
  qui donne la valeur à écrire sur le ruban de la machine dans le cas
  où on lit un 0 (resp. 1). Par exemple, si la machine lit un 0 sur le
  ruban, et que son état courant est 3, alors on effectue l'affectation
  suivante : \texttt{ruban[pos]=write1[3];}. Notez que les tableaux
  \texttt{write0} et \texttt{write1} ne contiennent que des 0 ou des 1.
\item Un tableau d'entiers \texttt{depl0} (resp. \texttt{depl1}) de
  taille 5 qui donne le déplacement de la tête de lecture avec -1 pour
  un déplacement vers la gauche et +1 pour un déplacement vers la
  droite. Par exemple, si la machine lit un 1 sur le ruban et que son
  état courant est 4, alors, la position de la tête de lecture est
  modifiée de la façon suivante : \texttt{pos=pos+depl1[4]}.
\item Un tableau d'entiers  \texttt{nextstate0} (resp.  \texttt{nextstate1)} de taille 5 qui donne
  la valeur du prochain état de la machine dans le cas où on lit un 0
  (resp. 1). Par exemple, si la machine lit un 1 sur le ruban et que
  son état courant est 2, alors, le prochain état sera l'entier
  contenu dans  \texttt{nextstate1[2]} (\texttt{cstate=nexstate1[2]}). Attention, il y a un élément du tableau
  \texttt{nextstate0} ou \texttt{nextstate1} qui est égal à -1 qui
  arrête la machine.\\
\end{itemize}
Pour tester l'algorithme, on prendra la machine de Turing suivante :
\begin{itemize}
\item \texttt{write0=\{1,1,1,1,1\}}
\item \texttt{write1=\{0,1,0,0,0\}}
\item \texttt{depl0=\{1,1,1,1,-1\}}
\item \texttt{depl1=\{-1,-1,-1,-1,1\}}
\item \texttt{nextstate0=\{1,2,3,4,3\}}
\item \texttt{nextstate1=\{-1,0,1,2,4\}}
\end{itemize}

\begin{solution}
%\begin{algorithm}[h!]
%\caption{D\'etermination d'un nombre d'occurences d'un entier dans une saisie}
%\label{algo_cm}
\begin{algorithmic}[1]
\REQUIRE{write0,write1,depl0,depl1,nexstate0,nexstate1 : tableaux d'entiers}
\ENSURE{compte : entier, nombre de déplacement}
\\
{\bf Donn\'ees locales : } pos,cstate : entiers ; ruban[] : tableau d'entiers
\\
\STATE pos $\leftarrow 50$
\STATE cstate $\leftarrow 0$
\STATE compte $\leftarrow 0$
\STATE Initialise ruban avec des zéros
\WHILE{cstate $\neq -1$}
\IF{ruban[pos] $=0$}
\STATE ruban[pos] $\leftarrow$ write0[cstate]
\STATE pos $\leftarrow$ pos + depl0[cstate]
\STATE cstate $\leftarrow$ nextstate0[cstate]
\ELSE
\STATE ruban[pos] $\leftarrow$ write1[cstate]
\STATE pos $\leftarrow$ pos + depl1[cstate]
\STATE cstate $\leftarrow$ nextstate1[cstate]
\ENDIF
\STATE compte=compte+1
\ENDWHILE
\end{algorithmic}
%\end{algorithm}
\end{solution}
